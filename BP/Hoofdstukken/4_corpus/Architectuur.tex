\chapter{Architectuur}
\label{ch:cassandra_architectuur}
% Peer to peer
% Cluster en snitches en partitioners
% Partitioneren via hashing (Lakhman 5.1)
%% Strong case against ByteOrderedPartitioner

%Structuur van Cassandra Data Modeling and Analysis
\section{Partitionering}
% DM p18 -- desc 5.1 -- HA p57
Een van de belangrijkste zaken bij Cassandra is het horizontaal schalen.
Om dit te kunnen doen moet de data dynamisch gepartitioneerd worden over de nodes van een cluster.
Dit wordt bij Cassandra voor elkaar gekregen door het gebruik van partitioners.
Deze partitioners maken gebruik van hash functies om te bepalen op welke node de data moet geplaatst worden.
Bij consistente hashing bij Cassandra wordt de output behandeld als een "ring".
Elke node in de cluster krijgt een willekeurige waarde toegewezen en deze waarde bepaald dan de plaats in de ring \citep{lakshman2010cassandra}.


Er zijn drie soorten partitioners in Cassandra:

\begin{enumerate}
	% p81
	\item \textbf{RandomPartitioner}:
	Dit was de standaard partitioner tot Cassandra 1.2.
	Via de MD5 hash van de rijsleutel probeert deze partitioner de data evenwichtig over alle nodes te verspreiden.
	Omdat de MD5 Hash functie vrij traag is, werd er een andere partitioner als standaard aangesteld.
	
	% p81
	\item \textbf{Murmur3Partitioner}:
	Sinds Cassandra 1.2 is dit de standaard partitioner van Cassandra.
	Deze partitioner maakt gebruik van de MurmurHash functie.
	De hash wordt op deze manier een 64-bit hashwaarde van de rijsleutel.
	
	% p82
	\item \textbf{ByteOrderedPartitioner}:
	Zoals de naam doet vermoeden wordt deze partitioner gebruikt voor geordende partitionering.
	Deze partitioner ordent de rijen volgens de bytes van de rijsleutel.
	De tokens worden berekend a.d.h.v. hexadecimale representatie van de leidende tekens van de rijsleutel.
	Op deze manier kan men, net zoals men met een cursor in een SQL onmgeving zou doen, de tabel geordend overlopen op primaire sleutel.
	
\end{enumerate}

\section{Replicatie}
% p19
% High availability p63
\section{Snitches}
% p20
\section{Seed node}
% p20
\section{Gossip en foutdetectie}
% p21
\section{Herstelmechanisme}
% %p24