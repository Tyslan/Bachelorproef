\chapter{Architectuur}
\label{ch:cassandra_architectuur}
% Peer to peer
% Cluster en snitches en partitioners
% Partitioneren via hashing (Lakhman 5.1)
%% Strong case against ByteOrderedPartitioner

%Structuur van Cassandra Data Modeling and Analysis
\section{Partitionering}
% DM p18 -- desc 5.1 -- HA p57
Een van de belangrijkste zaken bij Cassandra is het horizontaal schalen.
Om dit te kunnen doen moet de data dynamisch gepartitioneerd worden over de nodes van een cluster.
Dit wordt bij Cassandra voor elkaar gekregen door het gebruik van partitioners.
Deze partitioners maken gebruik van hash functies om te bepalen op welke node de data moet geplaatst worden.
Bij consistente hashing bij Cassandra wordt de output behandeld als een "ring".
Elke node in de cluster krijgt een willekeurige waarde toegewezen en deze waarde bepaald dan de plaats in de ring \citep{lakshman2010cassandra}.


Er zijn drie soorten partitioners in Cassandra:

\begin{enumerate}
	% p81
	\item \textbf{Murmur3Partitioner}:
	
	
	% p81
	\item \textbf{RandomPartitioner}:
	% p82
	\item \textbf{ByteOrderedPartitioner}:
\end{enumerate}

\section{Replicatie}
% p19
% High availability p63
\section{Snitches}
% p20
\section{Seed node}
% p20
\section{Gossip en foutdetectie}
% p21
\section{Herstelmechanisme}
% %p24