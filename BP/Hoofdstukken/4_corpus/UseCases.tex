\chapter{Use cases voor Cassandra}
\label{ch:cassandra_ucs}

\section{Messaging}
% http://www.planetcassandra.org/blog/functional_use_cases/messaging/
Voor messaging systemen is het zeer belangrijk dat er 100\% uptime is.
Door de peer-to-peer architectuur van Cassandra is er geen single point of failure.
Als hier ook nog eens multi-datacenter replicatie aan toegevoegd wordt is het bijna onmogelijk dat het systeem downtime kent \citep{Chan2014Messaging}.

Binnen deze use case zijn de belangrijkste gebruikers Instagram, New York Times, ComCast\dots

\section{Fraude bestrijding}
% http://www.planetcassandra.org/blog/functional_use_cases/fraud_detection/
Cassandra voorziet real-time monitoring.
Hiervoor kan nodetool of een ander tools zoals Datastax OpsCenter.
Op deze manier kan verdacht gedrag snel opgespoord worden.
Toch hoeft de fraude bestrijding niet enkel op deze manier te gebeuren.

\cite{Nguyen2014Fraud} geeft een mooi voorbeeld om spam mail op te sporen bij Nextgate.
Binnen de database wordt de data zo gemodelleerd dat alle berichten met eenzelfde inhoud in dezelfde rij terecht komen.
Door een simpele count uit te voeren op de rijen kan men nagaan hoe vaak eenzelfde bericht verstuurd is.
Als de count een bepaald getal overschrijdt dan wordt het bericht als spam beschouwd.

Naast Nextgate zijn er ook nog andere belangrijkste gebruikers zoals Barracuda Networks, ZoneFox, Iovation\dots

\section{Product catalogi en afspeellijsten}
% http://www.planetcassandra.org/blog/functional_use_cases/product-catalogs-playlists/
Door de enorme schaalbaarheid, de hoge performantie en de hoge beschikbaarheid is Cassandra uitstekend geschikt om product catalogi en afspeellijsten bij te houden.
Zowel product catalogi als afspeellijsten kunnen exponentieel groeien.

De bekenste gebruikers binnen deze use case zijn SoundCloud, Spotify, Reddit, Netflix\dots

Bij Reddit wordt Cassandra in veel zaken gebruikt maar de belangrijkste toepassingen zijn  het voting systeem, gelikete pagina's en gelikete subreddits \citep{Harvey2013Reddit}.

\section{Internet Of Things en sensor data}
% http://www.planetcassandra.org/blog/functional_use_cases/internet-of-things-sensor-data/
Door de hoge doorvoersnelheid is Cassandra uitermate geschikt voor het opslaan van sensor gegevens.
Het is dan ook vooral deze eigenschap die vaak de doorslag geeft om voor Cassandra te kiezen als men bij deze use case terecht komt.

Bij deze use case zijn de bekendste gebruikers Aeris, i2O, CERN, NASA\dots

\cite{Keller2013Nasa} legt uit waarom er binnen NASA met Cassandra gewerkt wordt.
Cassandra laat volgens hem toe om de data op een veel natuurlijker manier in te geven.
Ook is het mogelijk met Cassandra om zeer snel de resultaten van een query te zien.
Hier heeft hij het voorbeeld dat zeer snel alle informatie over een specifiek ip-adres op een specifieke tijd kan opvragen.
Een ander groot pluspunt voor Cassandra is de ingebouwde time-to-live.
Op die manier wordt data die niet meer relevant is automatisch verwijderd.
De hoge schrijfsnelheid is voor NASA ook belangrijk omdat deze organisatie dag en nacht veel verschillende feeds ontvangt.

\section{Aanbevelingen en personalisatie}
% http://www.planetcassandra.org/blog/functional_use_cases/recommendation-engine-personalization/
% http://planetcassandra.org/blog/interview/ebay-chooses-apache-cassandra-to-power-next-generation-recommendation-engine/
% http://planetcassandra.org/blog/interview/gaming-dev-platform-unity-powers-up-with-cassandra-migrates-away-from-mongodb-for-a-scalable-low-latency-solution/
% http://planetcassandra.org/blog/yahoo-goes-woohoo-for-apache-cassandra-cassandra-wins-yahoo-japan-nosql-evaluation-with-lowest-latency-and-highest-scalability/

\section{Retail}
% http://www.planetcassandra.org/blog/industry_use_cases/retail/

\section{Digitale media}
% http://www.planetcassandra.org/blog/industry_use_cases/media/

\section{Financiële services}
% http://www.planetcassandra.org/blog/industry_use_cases/finance/