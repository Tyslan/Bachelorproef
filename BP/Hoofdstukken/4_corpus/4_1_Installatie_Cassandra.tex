\chapter{Installatie Cassandra}
\label{ch:installatie_cassandra}

Voor de installatie van de Cassandra cluster werd gebruik gemaakt van virtuele machines en het OpsCenter van Datastax.
De cluster omgeving werd opgezet door middel van vagrant machines.
De installatie van Cassandra en het opzetten de cluster gebeurde met het OpsCenter van Datastax.
Er werd geopteerd om gebruik te maken van OpsCenter omdat dit een gemakkelijke manier is om snel een Cassandra cluster te bekomen.
Van het OpsCenter werd er voor de community edition v\textit{xxxxx} gekozen.
Hiermee komt Cassandra 2.1.11 geïnstalleerd.

Om te beginnen aan het opzetten van de van de clusters werd geopteerd om gebruik te maken van virtuele machines, die geconfigureerd werden met Vagrant.
De setup bestaat uit 1 master node waarop het OpsCenter runt en dan 3 slave nodes waar de uiteindelijke Cassandra database op komt te runnen.
Na de NAT router van Oracle Virtual Box werd een privaat netwerk opgezet zodanig deze machines met elkaar konden communiceren.
Hiervoor moest elke machine een ip-adres krijgen binnen het netwerk en ook de /etc/hosts aangepast worden.

Eenmaal de virtuele machines correct geconfigureerd waren, werd er overgegaan tot de eigenlijke installatie van Cassandra.
Zoals eerder vermeld werd hiervoor gebruik gemaakt van het OpsCenter.
Hiervoor werd op de master node naar de localhost:8888 gesurft om de installatie te starten. Op de pagina dit te voorschijn komt werd voor de optie 'brand new cluster gekozen'.