\chapter{Data importeren in Cassandra}
\label{ch:cassandra_import}

\section{Importeren via cqlsh}
Om de data via cqlsh te kunnen importeren dient eerst een keyspace aangemaakt te worden.
Bij het aanmaken van deze keyspace dient de replicatie strategie en de replicatie factor meegegeven te worden.
Na het aanmaken dient de tabel waarin de data zal worden geïmporteerd worden aangemaakt te worden.

Nu kunnen we via het "copy" commando, dat binnen cql voorzien is om data importeren in Cassandra.
Een aantal zaken dient hierbij opgemerkte te worden.

\begin{enumerate}
	\item De volgorde van de kolommen kan gespecificeerd worden aangezien Cassandra de kolommen automatisch alfabetisch sorteert en dit niet noodzakelijk het geval is bij het csv bestand.
	\item De scheidingsteken van de velden in het csv bestand kan gespecificeerd worden evenals de encapsulering van de velden.
	\item Er kan specifiek meegegeven worden wat Cassandra moet aanvangen met de null waarde.
\end{enumerate}

Dit is een zeer eenvoudige manier om data te importeren in Cassandra.
Toch wordt dit deze methode niet aangeraden om te gebruiken bij het importeren van grote hoeveelheden data.
Bij ca. 1 miljoen rijen, afhankelijk van het aantal kolommen, kan het zijn dat deze methode vast loopt.
Dit kan men op een aantal manieren oplossen.
De drie meest voorkomende zijn:

\begin{enumerate}
	\item Het opsplitsen van één groot csv bestand in verschillende kleinere bestanden.
	\item Het gebruik van de sstableloader die Cassandra voorziet.
	\item Het gebruik van cassandra-loader
\end{enumerate}

\section{Importeren via sstableloader}

Het importeren van data via sstableloader is eveneens een eenvoudig proces.
Hier zijn reeds verschillende implementaties van, zoals cassandra bulkloader.

Enkele belangrijke nadelen van deze manier zijn:
\begin{itemize}
	\item Men moet een aangepaste applicatie schrijven om dit te kunnen gebruiken.
	\item Om sstrableloader te kunnen gebruiken dienen alle nodes van de cluster online te zijn.
	\item De SSTable dient aangemaakt te zijn vooraleer men deze methode kan gebruiken.
\end{itemize}

\section{Importeren via cassandra-loader}
Dit is een Java programma van Brain Hess.
Dit programma maakt gebruik van de CQL driver die voorzien is door DataStax.
Het ganse principe van dit programma is om asynchroon cql inserts te doen.

\section{Een vergelijking van de drie methodes}
