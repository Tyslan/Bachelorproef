\chapter{Data importeren in Cassandra}
\label{ch:cassandra_import}

\section{Importeren via cqlsh}
Om de data via cqlsh te kunnen importeren dient eerst een keyspace aangemaakt te worden.
Bij het aanmaken van deze keyspace dient de replicatie strategie en de replicatie factor meegegeven te worden.
Na het aanmaken van de keyspace, dient hierin de tabel waarin de data geïmporteerd zal  worden aangemaakt te worden.

Nu kunnen we via het 'COPY' commando, dat binnen cql voorzien is om data te importeren in Cassandra \citep{Cannon2012Import}.
Een aantal zaken dienen hierbij opgemerkt te worden.

\begin{enumerate}
	\item De volgorde van de kolommen kan gespecificeerd worden aangezien Cassandra de kolommen automatisch alfabetisch sorteert en dit niet noodzakelijk het geval is bij het csv bestand.
	\item Het scheidingsteken van de velden in het csv bestand kan gespecificeerd worden evenals de encapsulering van de velden.
	\item Er kan specifiek meegegeven worden wat Cassandra moet aanvangen met de null waarde.
\end{enumerate}

Dit is een zeer eenvoudige manier om data te importeren in Cassandra.
Alle inserts gebeuren asynchroon, wat uiteraard goed is.
Toch wordt deze methode niet aangeraden om te gebruiken bij het importeren van grote hoeveelheden data.
Een eerste reden is dat er maar met één node connectie gemaakt wordt.
Hierdoor wordt de werklast niet evenwichtig verdeeld over alle nodes.
Bij ca. 1 miljoen rijen, afhankelijk van het aantal kolommen, kan het zijn dat deze methode vast loopt.

De bovenstaande problemen kan men op een aantal manieren oplossen.
De drie meest voorkomende zijn:

\begin{enumerate}
	\item Het opsplitsen van één groot csv bestand in verschillende kleinere bestanden.
	\item Het gebruik van de sstableloader die Cassandra voorziet.
	\item Het gebruik van cassandra-loader
\end{enumerate}

\section{Importeren via sstableloader}

Het importeren van data via sstableloader is eveneens een eenvoudig proces, maar hier komt wat meer werk aan te pas.

Enkele belangrijke nadelen van deze manier zijn:
\begin{itemize}
	\item Men moet een aangepaste applicatie schrijven om dit te kunnen gebruiken.
	\item Om sstableloader te kunnen gebruiken dienen alle nodes van de cluster online te zijn.
	\item De sstable dient aangemaakt te zijn vooraleer men deze methode kan gebruiken.
\end{itemize}

De reden waarom alle nodes online moeten zijn is omdat de hash van het record meteen bepaald wordt en dan doorgestuurd wordt naar de node waar het hoort te staan.

\section{Importeren via cassandra-loader}
Dit is een Java programma van Brain Hess.
Dit programma maakt gebruik van de CQL driver die beschikbaar gesteld wordt door DataStax.
Het ganse principe van dit programma is om asynchroon te werken en om met iedere node in de cluster te verbinden om zo het werk evenwichtig te verdelen.
Dit laatste was een van de struikelblokken bij het importeren van data via cqlsh.

\section{Testen van de verschillende loaders}
In dit stuk is het de bedoeling om de verschillende manieren van data import te vergelijken.
In deze vergelijking werd altijd dezelfde data gebruikt.
Enkel de hoeveelheid data die ingeladen werd was verschillend.

Eerst was de meest eenvoudige manier aan de beurt, het COPY commando via cqlsh.
Hier werden verschillende bestanden met dezelfde data, maar met een ander aantal rijen ingeladen.

\begin{table}[H]
	\centering
	\begin{tabular}{|r|r|}
		\hline
		Aantal rijen & Tijd (s) \\
		\hline
		\hline
		1 000 & 0.388 \\
		\hline
		10 000 & 2.068 \\
		\hline
		100 000 & 23.635 \\
		\hline
		1 000 000 & 195.742 \\
		\hline
		10 000 000 & 2 150.821\\
		\hline
	\end{tabular}
	\caption{Importeren van data met cqlsh}
	\label{tab:cas_cqlsh}
\end{table}

Als men tabel \ref{tab:cas_cqlsh} bekijkt ziet men hier dat er een lineair verband is tussen het aantal rijen en de tijd die nodig is om alle rijen te importeren.
Dit is ook wat verwacht werd.

Het testen met sstableloader werd niet gedaan omwille van de reden dat alle nodes online moeten zijn.
In een systeem waar alles ingezet wordt op een hoge beschikbaarheid, is het dan ook niet aangeraden om een methode te gebruiken waar import crasht als er een node offline gaat.

Het gebruik van cassandra-loader leverde veel problemen op.
Dit kwam deels door het gebruik van de virtuele machines.
De cassandra-loader verwacht 8 GB werkgeheugen.
Dit was oorspronkelijk niet beschikbaar op de virtuele machine.
Hierdoor moest het originele script aangepast worden.
Echter na de aanpassing van het virtueel geheugen dat toegewezen wordt aan de Java virtuele machine crashte het programma.

Doordat het inladen via cassandra-loader niet werkte kan er geen vergelijking gemaakt worden.
Toch moet er opgemerkt worden dat de manier van cassandra-loader voor het importeren en het importeren via cqlsh niet zo veel verschilt in theorie.
Cassandra-loader gaat er enkel nog voor zorgen dat het werk over alle online nodes evenwichtig verdeeld wordt.
