\chapter{Inleiding}
\label{ch:inleiding}

% De inleiding moet de lezer alle nodige informatie verschaffen om het onderwerp te begrijpen zonder nog externe werken te moeten raadplegen \citep{Pollefliet2011}. Dit is een doorlopende tekst die gebaseerd is op al wat je over het onderwerp gelezen hebt (literatuuronderzoek).

%Je verwijst bij elke bewering die je doet, vakterm die je introduceert, enz. naar je bronnen. In \LaTeX{} kan dat met het commando \texttt{$\backslash${cite\{\}}} of \texttt{$\backslash${citep\{\}}}. Als argument van het commando geef je de ``sleutel'' van een ``record'' in een bibliografische databank in het Bib\TeX{}-formaat (een tekstbestand). Als je expliciet naar de auteur verwijst in de zin, gebruik je \texttt{$\backslash${}cite\{\}}.
% Soms wil je de auteur niet expliciet vernoemen, dan gebruik je \texttt{$\backslash${}citep\{\}}. Hieronder een voorbeeld van elk.

%\cite{Knuth1998} schreef een van de standaardwerken over sorteer- en zoekalgoritmen. Experten zijn het erover eens dat cloud computing een interessante opportuniteit vormen, zowel voor gebruikers als voor dienstverleners op vlak van informatietechnologie~\citep{Creeger2009}.
\section{NoSQL}
Sinds de term in 1998 voor het eerst gebruikt werd door Carlo Strozzi, ging een grote bal aan het rollen rond NoSQL databanken.
Toen hij deze term de wereld instuurde, bedoelde Strozzi dat de databank die op dat moment besproken werd geen SQL interface aanbood.
In 2009 werd de term NoSQL opnieuw gebruikt door Johan Oskarsson als hashtag voor een meetup waar de problemen met relationele databanken en de huidige manier van programmeren besproken gingen worden.
Nu wordt NoSQL begrepen als 'Not only SQL', wat erop wijst dat er meerdere manieren zijn om data op te slaan. \citep{Fowler2013Introduction}

NoSQL databanken vinden hun oorsprong in de verschillen tussen een relationeel model en het object georiënteerd programmeren, het zogenoemde 'impedance mismatch', het feit dat relationele databanken vaak niet goed werken op een cluster of niet met grote hoeveelheden realtime data om kunnen gaan\dots
Hierdoor maken NoSQL databanken vaak geen gebruik van een relationeel model, zijn ze gemaakt om op clusters te werken, zijn ze schema-less\dots
Kortom databanken onder de noemer NoSQL zijn aangepast om de problemen, die zich nu binnen de Informatica manifesteren, aan te pakken \citep{Fowler2012NoSQLDef}.

\cite{Sadalage2014OverviewNoSQL} zegt dat binnen de NoSQL databanken vier grote types naar voor geschoven zijn, namelijk key-value stores (Riak, Redis\dots), document stores (MongoDB, CouchDB\dots), Column Family Stores (Cassandra, HBase\dots) en Graph databases (Neo4J, Infinite Graph\dots).
Elk van deze types heeft zijn eigen specifieke use cases.
Zelf binnen de verschillende types komen er nog verschillende use cases voor.

\section{Cassandra}

In het vervolg van deze bachelorproef wordt de focus gelegd op de NoSQL databank Cassandra.
Cassandra is een Column Family database die focust op schaalbaarheid en beschikbaarheid, zonder aan performantie in te boeten.
Cassandra is op dit moment een productiewaardige database.
Enkele bekende gebruikers van Cassandra zijn Facebook, Apple, Netflix, GitHub, Instagram, GoDaddy\dots

Cassandra is een project dat zijn oorsprong vond bij Facebook.
In 2008 was Cassandra, bedacht door Lakshman en Malik, de oplossing voor het Inbox Search probleem van Facebook.
De moeilijkheid hierbij was dat er een systeem nodig was die een hoge throughput nodig heeft, die miljarden write operaties per dag moet aankunnen en die kan mee schalen met het aantal gebruikers \citep{lakshman2010cassandra}.
Om tot deze oplossing te komen baseerden Lakshman en Malik zich op twee andere projecten.

Een eerste project waarop Cassandra gebaseerd is, is Google BigTable.
Google BigTable had namelijk al een oplossing voor een eerste probleem dat Lakshman en Malik moesten oplossen: een schaalbare database die geen realtime antwoorden opoffert \citep{chang2008bigtable}.
Lakshmans vorige project, Amazon Dynamo, was de tweede inspiratiebron voor Cassandra.
Dynamo zorgde eerder al voor een hoge betrouwbaarheid bij een schaalbare database \citep{decandia2007dynamo}.

Hoewel Cassandra niet verder uitgebreid moest worden omdat het nog steeds voldeed aan de voorwaarden om het probleem van Facebook op te lossen, is Cassandra verder blijven groeien sinds 2008.
Zo kan Cassandra nu bijvoorbeeld ook overweg met gestructureerde, semi-gestructureerde en ongestructureerde data \citep{kan2014cassandra}.

\section{Probleemstelling en Onderzoeksvragen}
\label{sec:onderzoeksvragen}

% TODO: Wees zo concreet mogelijk bij het formuleren van je
% onderzoeksvra(a)g(en). Een onderzoeksvraag is trouwens iets waar nog
% niemand op dit moment een antwoord heeft (voor zover je kan nagaan).

Cassandra belooft een groot aantal zaken en binnen deze bachelorproef is het de bedoeling om deze beloften na te gaan.
Eerst en vooral zal de schaalbaarheid van Cassandra gecontroleerd worden.
Is het werkelijk eenvoudig om de database op verschillende eenvoudige servers te installeren?
Een tweede punt dat nagegaan wordt is de betrouwbaarheid van Cassandra.
Is er werkelijk geen single point of failure en in hoeverre zijn back-ups nodig binnen deze omgeving?