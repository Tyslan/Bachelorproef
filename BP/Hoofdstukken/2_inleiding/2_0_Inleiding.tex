\chapter{Inleiding}
\label{ch:inleiding}

% De inleiding moet de lezer alle nodige informatie verschaffen om het onderwerp te begrijpen zonder nog externe werken te moeten raadplegen \citep{Pollefliet2011}. Dit is een doorlopende tekst die gebaseerd is op al wat je over het onderwerp gelezen hebt (literatuuronderzoek).

%Je verwijst bij elke bewering die je doet, vakterm die je introduceert, enz. naar je bronnen. In \LaTeX{} kan dat met het commando \texttt{$\backslash${cite\{\}}} of \texttt{$\backslash${citep\{\}}}. Als argument van het commando geef je de ``sleutel'' van een ``record'' in een bibliografische databank in het Bib\TeX{}-formaat (een tekstbestand). Als je expliciet naar de auteur verwijst in de zin, gebruik je \texttt{$\backslash${}cite\{\}}.
% Soms wil je de auteur niet expliciet vernoemen, dan gebruik je \texttt{$\backslash${}citep\{\}}. Hieronder een voorbeeld van elk.

%\cite{Knuth1998} schreef een van de standaardwerken over sorteer- en zoekalgoritmen. Experten zijn het erover eens dat cloud computing een interessante opportuniteit vormen, zowel voor gebruikers als voor dienstverleners op vlak van informatietechnologie~\citep{Creeger2009}.
\section{NoSQL}
Sinds de term voor het eerst in 1998 gebruikt werd door Carlo Strozzi, is er veel te doen rond NoSQL databanken.
Toen hij deze term de wereld instuurde bedoelde Strozzi dat de databank die op dat moment besproken werd geen SQL interface aanbood.
Dit wijkt af van wat nu onder NoSQL verstaan wordt.
NoSQL wordt nu begrepen als not only SQL, wat erop wijst dat er meerdere manieren zijn om data op te slaan.
Deze familie databanken maken 

NoSQL databanken vinden hun oorsprong in de verschillen tussen een relationeel model en het object geörienteerd programmeren, het zogenoemde "impedance mismatch", het feit dat relationele databanken vaak niet goed werken op een cluster of met grote hoeveelheden realtime data\dots
Hierdoor maken NoSQL databanken vaak geen gebruik van een relationeel model, zijn gemaakt om op clusters te werken.
Kortom ze zijn aangepast om de problemen die zich nu voordoen binnen de Informatica aan te pakken. \citep{Sadalage2014OverviewNoSQL}

Met de opkomst van NoSQL databanken zijn er 4 grote types naar voor gekomen, namelijk key-value stores (Riak, Redis...), document stores (MongoDB, CouchDB...), Column Family Stores (Cassandra, HBase...) en Graph databases (Neo4J, Infinite Graph...). 
Deze NoSQL databanken kwamen er in reactie op de problemen die men ondervond met relationele databanken. 

\section{Cassandra}

In het vervolg van deze bachelorproef wordt de focus gelegd op de NoSQL databank Cassandra.
% Wat geschiedenis van Cassandra.


\section{Probleemstelling en Onderzoeksvragen}
\label{sec:onderzoeksvragen}

% TODO: Wees zo concreet mogelijk bij het formuleren van je
% onderzoeksvra(a)g(en). Een onderzoeksvraag is trouwens iets waar nog
% niemand op dit moment een antwoord heeft (voor zover je kan nagaan).
