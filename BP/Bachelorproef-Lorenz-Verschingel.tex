%%========================================================================
%% LaTeX sjabloon voor stage/projectrapport of bachelorproef
%%  HoGent Bedrijf en Organisatie
%%========================================================================

%%========================================================================
%% Preamble
%%========================================================================

\documentclass[pdftex,a4paper,12pt]{report}

% XXX: Let op: dit sjabloon is gemaakt om dubbelzijdig af te drukken
% Voor enkelzijdig, verwijder ``twoside'' hierboven.

%%---------- Extra functionaliteit ---------------------------------------

\usepackage[utf8]{inputenc}  % Accenten gebruiken in tekst (vb. é ipv \'e)
\usepackage{amsfonts}        % extra wiskundige
\usepackage{amsmath}         %   symbolen (o.a. getallen-
\usepackage{amssymb}         %   verzamelingen N, R, Z, Q, etc.)
\usepackage[dutch]{babel}    % Taalinstellingen: woordsplitsingen,
                             %  commando's voor speciale karakters
                             %  ("dutch" voor NL)
\usepackage{eurosym}         % Euro-symbool €
\usepackage{geometry}
\usepackage{graphicx}        % Invoegen van tekeningen
\usepackage{subcaption}		 % invoegen van subfiguren
\usepackage{float}			 % Positie van figuren vastleggen
\usepackage[pdftex,bookmarks=true]{hyperref}
                             % klikbare links & verwijzingen,
\usepackage{parskip}		 % Paragrafen zonder inspringen, maar witruimte
                             %  inhoudstafel
\usepackage{listings}        % Broncode mooi opmaken
\usepackage{multirow}        % Tekst over verschillende cellen in tabellen
\usepackage{rotating}        % Tabellen en figuren roteren
\usepackage{natbib}          % Betere bibliografiestijlen
\usepackage{fancyhdr}        % Pagina-opmaak met hoofd- en voettekst

\usepackage[T1]{fontenc}     % Ivm lettertypes
\usepackage{lmodern}
\usepackage{textcomp}

\usepackage{lipsum}          % Voor vultekst (lorem ipsum)

%%---------- Layout ------------------------------------------------------

% hoofdingen, enz.
\pagestyle{fancy}
% enkel hoofdstuktitel in hoofding, geen sectietitel (vermijd overlap)
\renewcommand{\sectionmark}[1]{}

% lijn, wordt gebruikt in titelpagina
\newcommand{\HRule}{\rule{\linewidth}{0.5mm}}

% Leeg blad
\newcommand{\emptypage}{
\newpage
\thispagestyle{empty}
\mbox{}
\newpage
}

% Gebruik een schreefloos lettertype ipv het "oubollig" uitziende
% Computer Modern
\renewcommand{\familydefault}{\sfdefault}

% Commando voor invoegen Java-broncodebestanden (dank aan Niels Corneille)
% Gebruik: \codefragment{source/MijnKlasse.java}{Uitleg bij de code}
\newcommand{\codefragment}[2]{ \lstset{%
  language=java,
  breaklines=true,
  float=th,
  caption={#2},
  basicstyle=\scriptsize,
  frame=single,
  extendedchars=\true
}
\lstinputlisting{#1}}

%%---------- Documenteigenschappen ---------------------------------------
%% Vul dit aan met je eigen info:

% Je eigen naam
\newcommand{\student}{Lorenz Verschingel}

% De naam van je lector, begeleider, promotor
\newcommand{\promotor}{Sabine De Vreese}

% De naam van je co-promotor
\newcommand{\copromotor}{Jean-Jacques De Clercq}

% Indien je bachelorproef in opdracht van een bedrijf of organisatie
% geschreven is, geef je hier de naam.
\newcommand{\instelling}{HoGent}

% De titel van het rapport/bachelorproef
\newcommand{\titel}{NoSQL: Apache Cassandra}

% Datum van indienen
%TODO
\newcommand{\datum}{29 mei 2015}

% Faculteit
\newcommand{\faculteit}{Faculteit Bedrijf en Organisatie}

% Soort rapport
\newcommand{\rapporttype}{Scriptie voorgedragen tot het bekomen van de graad van\\Bachelor in de toegepaste informatica}

% Academiejaar
\newcommand{\academiejaar}{2015-2016}

% Examenperiode
%  - 1e semester = 1e examenperiode
%  - 2e semester = 2e examenperiode
%  - tweede zit = 3e examenperiode
\newcommand{\examenperiode}{Tweede examenperiode}

%%========================================================================
%% Inhoud document
%%========================================================================

\begin{document}

%%---------- Front matter ------------------------------------------------
%% Het voorblad - Hier moet je in principe niets wijzigen.

\begin{titlepage}
  \newgeometry{top=2cm,bottom=1.5cm,left=1.5cm,right=1.5cm}
  \begin{center}

    \begingroup
    \rmfamily
    \includegraphics[width=2.5cm]{img/HG-beeldmerk-woordmerk}\\[.5cm]
    \faculteit\\[3cm]
    \titel
    \vfill
    \student\\[3.5cm]
    \rapporttype\\[2cm]
    Promotor:\\
    \promotor\\
    Co-promotor:\\
    \copromotor\\[2.5cm]
    Instelling: \instelling\\[.5cm]
    Academiejaar: \academiejaar\\[.5cm]
    \examenperiode
    \endgroup

  \end{center}
  \restoregeometry
\end{titlepage}

% Schutblad

\emptypage


\begin{titlepage}
  \newgeometry{top=5.35cm,bottom=1.5cm,left=1.5cm,right=1.5cm}
  \begin{center}

    \begingroup
    \rmfamily
    \faculteit\\[3cm]
    \titel
    \vfill
    \student\\[3.5cm]
    \rapporttype\\[2cm]
    Promotor:\\
    \promotor\\
    Co-promotor:\\
    \copromotor\\[2.5cm]
    Instelling: \instelling\\[.5cm]
    Academiejaar: \academiejaar\\[.5cm]
    \examenperiode
    \endgroup

  \end{center}
  \restoregeometry
\end{titlepage}

\begin{abstract}
% TODO: De "abstract" of samenvatting is een kernachtige (max 1 blz. voor een
% thesis) synthese van het document. In ons geval beschrijf je kort de
% probleemstelling en de context, de onderzoeksvragen, de aanpak en de
% resultaten.
In de laatste jaren zijn de NoSQL databases aan een enorme opmars bezig.
Hierbij stellen bepaalde van deze databanken hun gebruiksgemak en kunnen nogal vaak zeer positief voor.

In deze bachelorproef wordt onderzoek gedaan naar de schaalbaarheid en de betrouwbaarheid van de NoSQL database Cassandra.
Dit onderzoek wordt uitgevoerd aan de hand van een virtuele cluster die opgezet is met Vagrant.
Een ander punt dat onderzocht werd, was hoe datamodellering in Cassandra in zijn werk gaat.

De schaalbaarheid wordt nagegaan door te kijken hoe gemakkelijk het is om een cluster op te zetten en om hieraan later nodes toe te voegen en weer te verwijderen.
In een virtuele omgeving is dit zeer gemakkelijk na te gaan.
Hierbij werd vastgesteld dat het opzetten van de cluster, het toevoegen van nodes en het verwijderen van nodes niet zo gemakkelijk gaat als wordt voorgesteld als men Apache Cassandra gebruikt.
Als men hier echter hulpprogramma's voor gaat gebruiken zoals het OpsCenter van DataStax gaat dit wel zeer vlot.

Om de betrouwbaarheid na te gaan werden opzettelijk nodes uitgezet om te kijken hoe Cassandra hierop reageert.
In deze testen doet Cassandra exact wat beloofd wordt.
De data blijft beschikbaar en wijzigingen worden bij het online komen van de node doorgegeven.

Bij de datamodellering werden enkele eigenaardigheden vastgesteld zoals het feit dat de primaire sleutel restricties oplegt aan de WHERE clausule.
Hierdoor moet men de data binnen Cassandra modelleren naar de query's die uitgevoerd zullen worden.

\end{abstract}

\chapter*{Voorwoord}
\label{ch:voorwoord}

% TODO: Vergeet ook niet te bedanken wie je geholpen/gesteund/... heeft
Deze bachelorproef markeert het einde van mijn driejarige opleiding Toegepaste Informatica aan de Hogeschool Gent.
Ik heb dit onderwerp, de NoSQL databank Cassandra, gekozen omdat ik er tijdens mijn stage mee aan de slag moest en deze technologie me enorm aansprak.

Het schrijven van een bachelorproef neemt veel tijd in beslag en zonder de hulp van heel wat mensen zou ik dit veel moeilijker tot een goed einde gebracht hebben.
Aan hen wil ik mijn grote dank uitdrukken.

Eerst en vooral wil ik de universiteit Gent bedanken voor het aanbieden van de data en in het bijzonder mijn co-promotor, de heer Jean-Jacques De Clercq, voor de technische ondersteuning en het aanhalen van het bedrijf DataStax.
Ik wil hem ook bedanken voor het sturen van het onderwerp in de richting van de mogelijkheden van Cassandra.

Verder wil ik ook mevrouw Sabine De Vreese, de promotor van deze bachelorproef bedanken voor de ondersteuning bij het opstellen van de planning en de opvolging vanuit de school.

Ten slotte wil ik ook nog mijn familie en vrienden bedanken voor de hulp bij het nalezen van mij bachelorproef, de morele ondersteuning, het eenvoudig formuleren van bepaalde ingewikkelde technische aspecten\dots

Tijdens deze bachelorproef heb ik veel bijgeleerd en ik hoop dat ook anderen informatie uit deze bachelorproef zullen kunnen halen.

\vspace{4 em}

\begin{flushright}
	\textit{Lorenz Verschingel}	
\end{flushright}
\begin{flushright}
	\textit{Student Toegepaste Informatica}
\end{flushright}
\begin{flushright}
	\textit{2016}
\end{flushright}

%%---------- Kern --------------------------------------------------------

\chapter{Inleiding}
\label{ch:inleiding}

% De inleiding moet de lezer alle nodige informatie verschaffen om het onderwerp te begrijpen zonder nog externe werken te moeten raadplegen \citep{Pollefliet2011}. Dit is een doorlopende tekst die gebaseerd is op al wat je over het onderwerp gelezen hebt (literatuuronderzoek).

%Je verwijst bij elke bewering die je doet, vakterm die je introduceert, enz. naar je bronnen. In \LaTeX{} kan dat met het commando \texttt{$\backslash${cite\{\}}} of \texttt{$\backslash${citep\{\}}}. Als argument van het commando geef je de ``sleutel'' van een ``record'' in een bibliografische databank in het Bib\TeX{}-formaat (een tekstbestand). Als je expliciet naar de auteur verwijst in de zin, gebruik je \texttt{$\backslash${}cite\{\}}.
% Soms wil je de auteur niet expliciet vernoemen, dan gebruik je \texttt{$\backslash${}citep\{\}}. Hieronder een voorbeeld van elk.

%\cite{Knuth1998} schreef een van de standaardwerken over sorteer- en zoekalgoritmen. Experten zijn het erover eens dat cloud computing een interessante opportuniteit vormen, zowel voor gebruikers als voor dienstverleners op vlak van informatietechnologie~\citep{Creeger2009}.
\section{NoSQL}
Sinds de term in 1998 voor het eerst gebruikt werd door Carlo Strozzi, ging een grote bal aan het rollen rond NoSQL databanken.
Toen hij deze term de wereld instuurde, bedoelde Strozzi dat de databank die op dat moment besproken werd geen SQL interface aanbood.
In 2009 werd de term NoSQL opnieuw gebruikt door Johan Oskarsson als hashtag voor een meetup waar de problemen met relationele databanken en de huidige manier van programmeren besproken gingen worden.
Nu wordt NoSQL begrepen als 'Not only SQL', wat erop wijst dat er meerdere manieren zijn om data op te slaan. \citep{Fowler2013Introduction}

NoSQL databanken vinden hun oorsprong in de verschillen tussen een relationeel model en het object georiënteerd programmeren, het zogenoemde 'impedance mismatch', het feit dat relationele databanken vaak niet goed werken op een cluster of niet met grote hoeveelheden realtime data om kunnen gaan\dots
Hierdoor maken NoSQL databanken vaak geen gebruik van een relationeel model, zijn ze gemaakt om op clusters te werken, zijn ze schema-less\dots
Kortom databanken onder de noemer NoSQL zijn aangepast om de problemen, die zich nu binnen de Informatica manifesteren, aan te pakken \citep{Fowler2012NoSQLDef}.

\cite{Sadalage2014OverviewNoSQL} zegt dat binnen de NoSQL databanken vier grote types naar voor geschoven zijn, namelijk key-value stores (Riak, Redis\dots), document stores (MongoDB, CouchDB\dots), Column Family Stores (Cassandra, HBase\dots) en Graph databases (Neo4J, Infinite Graph\dots).
Elk van deze types heeft zijn eigen specifieke use cases.
Zelf binnen de verschillende types komen er nog verschillende use cases voor.

\section{Cassandra}

In het vervolg van deze bachelorproef wordt de focus gelegd op de NoSQL databank Cassandra.
Cassandra is een Column Family database die focust op schaalbaarheid en beschikbaarheid, zonder aan performantie in te boeten.
Cassandra is op dit moment een productiewaardige database.
Enkele bekende gebruikers van Cassandra zijn Facebook, Apple, Netflix, GitHub, Instagram, GoDaddy\dots

Cassandra is een project dat zijn oorsprong vond bij Facebook.
In 2008 was Cassandra, bedacht door Lakshman en Malik, de oplossing voor het Inbox Search probleem van Facebook.
De moeilijkheid hierbij was dat er een systeem nodig was die een hoge throughput nodig heeft, die miljarden write operaties per dag moet aankunnen en die kan mee schalen met het aantal gebruikers \citep{lakshman2010cassandra}.
Om tot deze oplossing te komen baseerden Lakshman en Malik zich op twee andere projecten.

Een eerste project waarop Cassandra gebaseerd is, is Google BigTable.
Google BigTable had namelijk al een oplossing voor een eerste probleem dat Lakshman en Malik moesten oplossen: een schaalbare database die geen realtime antwoorden opoffert \citep{chang2008bigtable}.
Lakshmans vorige project, Amazon Dynamo, was de tweede inspiratiebron voor Cassandra.
Dynamo zorgde eerder al voor een hoge betrouwbaarheid bij een schaalbare database \citep{decandia2007dynamo}.

Hoewel Cassandra niet verder uitgebreid moest worden omdat het nog steeds voldeed aan de voorwaarden om het probleem van Facebook op te lossen, is Cassandra verder blijven groeien sinds 2008.
Zo kan Cassandra nu bijvoorbeeld ook overweg met gestructureerde, semi-gestructureerde en ongestructureerde data \citep{kan2014cassandra}.

\section{Probleemstelling en Onderzoeksvragen}
\label{sec:onderzoeksvragen}

% TODO: Wees zo concreet mogelijk bij het formuleren van je
% onderzoeksvra(a)g(en). Een onderzoeksvraag is trouwens iets waar nog
% niemand op dit moment een antwoord heeft (voor zover je kan nagaan).

Cassandra belooft een groot aantal zaken en binnen deze bachelorproef is het de bedoeling om deze beloften na te gaan.
Eerst en vooral zal de schaalbaarheid van Cassandra gecontroleerd worden.
Is het werkelijk eenvoudig om de database op verschillende eenvoudige servers te installeren?
Een tweede punt dat nagegaan wordt is de betrouwbaarheid van Cassandra.
Is er werkelijk geen single point of failure en in hoeverre zijn back-ups nodig binnen deze omgeving?

\chapter{Methodologie}
\label{ch:methodologie}

% TODO: Hoe ben je te werk gegaan? Verdeel je onderzoek in grote fasen, en
% licht in elke fase toe welke stappen je gevolgd hebt. Verantwoord waarom je
% op deze manier te werk gegaan bent. Je moet kunnen aantonen dat je de best
% mogelijke manier toegepast hebt om een antwoord te vinden op de
% onderzoeksvraag.

Om een antwoord te bieden op alle onderzoeksvragen werd deze bachelorproef opgesplitst in twee luiken.
Het eerste luik omvat het eerder theoretisch gedeelte, waar een literatuurstudie aan te pas kwam.
Het tweede luik omvat het praktisch gedeelte die nodig was om op een aantal vragen een antwoord te krijgen.

In het theoretisch gedeelte komt zoals eerder vermeld de literatuurstudie aan bod.
Hierin wordt dieper in gegaan op de architectuur, hoe data opgeslagen wordt binnen Cassandra, wat er juist bedoeld wordt met het meervoudig opslaan van data, hoe belangrijk back-ups zijn binnen dit systeem en voor welke problemen Cassandra een oplossing biedt\dots

Om dit alles te kunnen nagaan werd het praktisch gedeelte opgezet.
Eerst moest er een Cassandra cluster opgezet worden.
Dit gebeurde aan de hand van Vagrant virtuele machines.
Er werd geopteerd voor Vagrant omdat dit een snelle manier is om verschillende identieke virtuele machines op te zetten.
Ook kon aan de hand van één enkel script de volledige omgeving gecontroleerd worden.
Voor de installatie van Cassandra werd eerst gekozen om met de apache versie te werken.
Het idee hiervan was om met de meest recente versie te werken.
Om praktische reden werd later verkozen om via het OpsCenter Community Edition van Datastax te werken.
Deze tool maakte het mogelijk om via een webinterface de databank Cassandra te beheren en te monitoren.
Door gebruik te maken van deze opzet konden snel nodes toegevoegd of verwijderd worden binnen de cluster, via deze opzet kon de schaalbaarheid makkelijk getest worden.

Toen deze cluster opgezet was, werd de data die voorzien werd door de Universiteit van Gent ingeladen in deze virtuele cluster.
Voor deze data ingeladen kon worden moest eerst stilgestaan worden bij het datamodel van deze data.
Deze data werd dus ook gebruikt om uit te leggen hoe je het best een datamodel opstelt binnen Cassandra.
Hier werd eveneens kort stil gestaan bij het verschil tussen datamodellering binnen een relationele databank en Cassandra. 

In een laatste deel moest ook nog de betrouwbaarheid van Cassandra getest worden.
Hier werd doelbewust een van de virtuele machines, een van de nodes van de databank, uitgeschakeld om te zien hoe Cassandra hierop reageert.
Doordat dit in een virtuele omgeving gebeurde is er geen risico op verlies van kritieke data.

%% TODO: de structuur en titel van deze hoofdstukken hangen af van je
% eigen onderzoek. Elke fase in je onderzoek kan een eigen hoofdstuk krijgen. Kies telkens een gepaste titel. ``Corpus'' is *GEEN* gepaste titel

\chapter{Data opslag in Cassandra}
\label{ch:cassandra_data}


\chapter{Opzetten van de Cassandra cluster}
\label{ch:cassandra_cluster}

\section{Apache Cassandra}
Om te beginnen aan het opzetten van de van de clusters werd geopteerd om gebruik te maken van virtuele machines, die geconfigureerd werden met Vagrant.

In een eerste poging om een werkende Cassandra cluster te bekomen werd er op elke Vagrant machine Cassandra 3.3, op moment van schrijven de meest recente versie, geïnstalleerd.
Nadat dit was gebeurd dienden nog enkele stappen te voltooid worden \citep{DataStax2016}.
Deze configuratie op deze manier gaf echter veel problemen, Cassandra werd telkens na enkele bewerkingen onbruikbaar met de foutboodschap "could not access pidfile for Cassandra".
Een eerste oplossing voor dit probleem was om te zorgen dat de user cassandra toegang had tot de pidfile, want namelijk niet het geval was doordat de installatie van Cassandra werd uitgevoerd door de vagrant setup.
Maar ook dit leverde weinig resultaat op.

\section{DataStax OpsCenter}

Uiteindelijk werd er geopteerd om gebruik te maken van OpsCenter omdat dit een gemakkelijke manier is om snel een Cassandra cluster te bekomen en omdat dit ook goede mogelijkheden tot monitoren van de database voorziet.
Van het OpsCenter werd er voor de community edition 5.2.4 gekozen.
Hiermee komt Cassandra 2.1.11 geïnstalleerd \citep{Cantoni2016}.

De setup bestaat uit 1 master node waarop het OpsCenter runt en dan 3 slave nodes waar de uiteindelijke Cassandra database op komt te runnen.
Na de NAT router van Oracle Virtual Box werd een privaat netwerk opgezet zodanig deze machines met elkaar konden communiceren.
Hiervoor moest elke machine een ip-adres krijgen binnen het netwerk en ook de /etc/hosts aangepast worden.

Eenmaal de virtuele machines correct geconfigureerd waren, werd er overgegaan tot de eigenlijke installatie van Cassandra.
Zoals eerder vermeld werd hiervoor gebruik gemaakt van het OpsCenter.
Hiervoor werd op de master node naar de localhost:8888 gesurft om de installatie te starten. Op de pagina dit te voorschijn komt werd voor de optie 'brand new cluster gekozen'.

In het volgende venster wordt er om verschillende zaken gevraagd.
Tabel \ref{tab:cas_conf} en figuur \ref{fig:cas_conf_1} geven weer hoe dit venster ingevuld werd.

\begin{table}[H]
  \begin{tabular}{|l|l|}
  \hline
  Property Name & Waarde \\
  \hline
  \hline
  Cluster Name & BP Cluster \\
  \hline
  Type & local \\
  \hline
  Package & datastax community 2.1.11 \\
  \hline
  Enpoint Snitch & GossipingPropertyFileSnitch \\
  \hline
  Username en password & vagrant/vagrant\\
  \hline
  Local Node Credentials & cassandra-node-1, cassandra-node-2, cassandra-node-3 \\
  \hline
  \end{tabular}
  \caption{Configuratie van de Cassandra Cluster}
  \label{tab:cas_conf}
\end{table}

\begin{figure}[H]
  	\centering
    \includegraphics[width=0.5\textwidth]{img/4_1_installatie_cassandra/1_Configuration_part_1}
    \caption{Cassandra: Instellingen deel 1}
    \label{fig:cas_conf_1}
\end{figure}

Hier dient ook een datacenter toegevoegd te worden.
Hierbij wordt de naam van het datacenter vrij gekozen en zijn de node properties het ip-adres van de slave nodes (Figuur: \ref{fig:cas_conf_2}).

\begin{figure}[H]
  	\centering
    \includegraphics[width=0.5\textwidth]{img/4_1_installatie_cassandra/1_Configuration_part_2}
    \caption{Cassandra: Instellingen deel 2}
    \label{fig:cas_conf_2}
\end{figure}

Eenmaal de datacenters zijn toegevoegd kan men verdergaan.
Bij het drukken op de knop 'build cluster word nog gevraagd om de fingerprints van de nodes te accepteren (Figuur: \ref{fig:cas_conf_3}).

\begin{figure}[H]
  	\centering
    \includegraphics[width=0.5\textwidth]{img/4_1_installatie_cassandra/1_Configuration_part_3}
    \caption{Cassandra: Instellingen deel 3}
    \label{fig:cas_conf_3}
\end{figure}

Hierna begint OpsCenter met de installatie van de Cassandra cluster (Figuur: \ref{fig:cas_install}).
Deze installatie neemt enkele ogenblikken in beslag.
Als hier fouten voorkomen ligt dit veelal aan het feit dat er onvoldoende werkgeheugen aanwezig is op de slave nodes.
In de setup die hier gebruikt werd het minimium aanvaarde geheugen geven aan de slave nodes, nl 2GB.

\begin{figure}[H]
	\centering
	\begin{subfigure}{.49\textwidth}
  		\centering
  		\includegraphics[width=.9\linewidth]{img/4_1_installatie_cassandra/1_Configuration_part_4}
  		\caption{Deel 1}
	\end{subfigure}
	\begin{subfigure}{.49\textwidth}
  		\centering
  		\includegraphics[width=.9\linewidth]{img/4_1_installatie_cassandra/1_Configuration_part_5}
  		\caption{Deel 2}
	\end{subfigure}
	\begin{subfigure}{.49\textwidth}
  		\centering
  		\includegraphics[width=.9\linewidth]{img/4_1_installatie_cassandra/1_Configuration_part_6}
  		\caption{Deel 3}
	\end{subfigure}
	\caption{Installatie van Cassandra door OpsCenter}
	\label{fig:cas_install}
\end{figure}
\chapter{Data modellering in Cassandra}
\label{ch:cassandra_modelling}


\chapter{Data importeren in Cassandra}
\label{ch:cassandra_import}

\section{Importeren via cqlsh}
Om de data via cqlsh te kunnen importeren dient eerst een keyspace aangemaakt te worden.
Bij het aanmaken van deze keyspace dient de replicatie strategie en de replicatie factor meegegeven te worden.
Na het aanmaken van de keyspace, dient men hierin de tabel waarin de data geïmporteerd zal worden, aan te maken.

Nu kunnen we door middel van het 'COPY' commando, dat binnen cql aangeboden wordt data importeren in Cassandra \citep{Cannon2012Import}.
%TODO copy commando invoeren

Een aantal zaken dienen hierbij opgemerkt te worden.

\begin{enumerate}
	\item De volgorde van de kolommen kan gespecificeerd worden aangezien Cassandra de kolommen automatisch alfabetisch sorteert en dit niet noodzakelijk het geval is bij het csv bestand.
	\item Het scheidingsteken van de velden in het csv bestand kan gespecificeerd worden evenals de encapsulering van de velden.
	\item Er kan specifiek meegegeven worden wat Cassandra moet aanvangen met de null waarde.
\end{enumerate}

Dit is een zeer eenvoudige manier om data te importeren in Cassandra.
Alle inserts gebeuren asynchroon, wat uiteraard goed is.
Toch wordt deze methode niet aangeraden om te gebruiken bij het importeren van grote hoeveelheden data.
Een eerste reden is dat er maar met één node connectie gemaakt wordt.
Hierdoor wordt de werklast niet evenwichtig verdeeld over alle nodes.
Bij ca. 1 miljoen rijen, afhankelijk van het aantal kolommen, kan het zijn dat deze methode vastloopt.

De bovenstaande problemen kan men op een aantal manieren oplossen.
De drie meest voorkomende zijn:

\begin{enumerate}
	\item Het opsplitsen van één groot csv bestand in verschillende kleinere bestanden.
	\item Het gebruik van de sstableloader die Cassandra voorziet.
	\item Het gebruik van cassandra-loader
\end{enumerate}

\section{Importeren met sstableloader}

Het importeren van data met sstableloader is eveneens een eenvoudig proces, maar hier komt wat meer werk aan te pas.

Enkele belangrijke nadelen van deze manier van werken zijn:
\begin{itemize}
	\item Men moet een aangepaste applicatie schrijven om dit te kunnen gebruiken.
	\item Om sstableloader te kunnen gebruiken dienen alle nodes van de cluster online te zijn.
	\item De sstable dient aangemaakt te zijn vooraleer men deze methode kan gebruiken.
\end{itemize}

De reden waarom alle nodes online moeten zijn is omdat de hash van het record meteen bepaald en daarna doorgestuurd wordt naar de node waarop de data hoort te staan.

\section{Importeren met cassandra-loader}
%TODO referentie
Dit is een Java programma van Brian Hess.
Dit programma maakt gebruik van de CQL driver, die beschikbaar gesteld wordt door DataStax.
Het ganse principe van dit programma is om asynchroon te werken en om met iedere node in de cluster verbinding te maken.
Op deze manier kan het werk evenwichtig verdeeld worden.
Dit laatste was één van de struikelblokken bij het importeren van data via cqlsh.

\section{Testen van de verschillende loaders}
In dit stuk is het de bedoeling om de verschillende manieren van data import te vergelijken.
In deze vergelijking werd altijd dezelfde data gebruikt.
Enkel het aantal records dat ingeladen werd, was verschillend.

Eerst was de meest eenvoudige manier aan de beurt, het COPY commando via cqlsh.
Hier werden verschillende bestanden met dezelfde data, maar met een ander aantal rijen ingeladen.

\begin{table}[H]
	\centering
	\begin{tabular}{|r|r|}
		\hline
		Aantal rijen & Tijd (s) \\
		\hline
		\hline
		1 000 & 0.388 \\
		\hline
		10 000 & 2.068 \\
		\hline
		100 000 & 23.635 \\
		\hline
		1 000 000 & 195.742 \\
		\hline
		10 000 000 & 2 150.821\\
		\hline
	\end{tabular}
	\caption{Importeren van data met cqlsh}
	\label{tab:cas_cqlsh}
\end{table}

Als men tabel \ref{tab:cas_cqlsh} bekijkt ziet men dat hier een lineair verband bestaat tussen het aantal rijen en de tijd die nodig is om alle rijen te importeren.
Dit leverde de verwachte resultaten op.

Een test met sstableloader werd niet uitgevoerd omdat daarvoor alle nodes online moeten zijn.
In een systeem waar alles ingezet wordt op hoge beschikbaarheid, is het dan ook niet aangeraden om een methode te gebruiken waar een import sowieso crasht als er een node offline gaat.

Het gebruik van cassandra-loader leverde veel problemen op.
Dit kwam deels door virtuele machines te gebruiken.
De cassandra-loader verwacht namelijk 8 GB werkgeheugen.
Op de virtuele machine was een dergelijke hoeveelheid werkgeheugen echter niet voorhanden.
Hierdoor moest het originele script lichtjes aangepast worden.
Na de aanpassing van het virtueel geheugen dat toegewezen wordt aan de Java virtuele machine, crashte het programma.

Doordat het inladen met cassandra-loader niet werkte kan er dus ook geen vergelijking tussen beide methodes gemaakt worden.
Er dient opgemerkt te worden dat beide manieren (cassandra-loader en importeren via cqlsh) in theorie niet zoveel van elkaar verschillen.
Het enige pluspunt dat cassandra-loader heeft, is dat deze tool het werk evenwichtig over alle online nodes gaat verdelen.
\chapter{Gedrag bij uitvallen van een node}
\label{ch:cassandra_uitval}

\section{Gossip en foutdetectie}
De nodes van Cassandra communiceren ongeveer iedere seconde met elkaar om informatie over hun eigen status en over de status van andere nodes waarmee ze in contact staan.
Hiervoor maakt Cassandra gebruik van een peer-to-peer protocol.
Op deze manier kunnen nodes snel de status van andere nodes weten.
Deze informatie wordt ook lokaal opgeslagen.

Cassandra gebruikt het "Phi Accrual Failure Detection Algoritm" voor het detecteren van het uitvallen van nodes \citep{kan2014cassandra}.
Het idee bij dit algoritme is dat de status niet weergegeven wordt door een booleaanse waarde, dood of levend, maar door een waarde die aangeeft hoe groot de kans is dat een node dood of leven is.
Als een node op deze manier dood verklaard wordt, blijven de andere nodes toch nog communiceren met deze node om ze te bepalen wanneer deze terug levend is.

\section{Herstelmechanisme}
% DM p24
% HA p70 p74

\section{Ondervindingen bij het uitzetten van een node}

\chapter{Conclusie}
\label{ch:conclusie}

% TODO: Trek een duidelijke conclusie, in de vorm van een antwoord op de
% onderzoeksvra(a)g(en). Reflecteer kritisch over het resultaat. Zijn er
% zaken die nog niet duidelijk zijn? Heeft het ondezoek geleid tot nieuwe
% vragen die uitnodigen tot verder onderzoek?

De schaalbaarheid van Cassandra werd nagegaan door nodes toe te voegen aan de cluster en het verwijderen hiervan.
Via de configuratie files van Cassandra is dit een heel karwei om te doen.
Verder werd er ook niet in geslaagd om op deze manier een werkende cluster te verkrijgen binnen deze bachelorproef.
Toen werd overgestapt naar het OpsCenter van DataStax was dit echter een ander verhaal.
Via deze weg was het zeer eenvoudig om een cluster te beheren en hier nodes aan toe te voegen of nodes te verwijderen.

Na de schaalbaarheid werd de betrouwbaarheid van de database getest.
Dit werd gedaan a.d.h.v. het uitschakelen van de nodes en zo af te toetsen of de theorie wel strookt met de werkelijkheid.
Bij uitvallen van de nodes kon telkens vastgesteld worden dat de data beschikbaar bleef.
Het hinting systeem van Cassandra bleek ook uitstekend te werken in deze test.
Toen enkel één node online was werden hier toch updates van records op uitgevoerd en bij het opnieuw opstarten van een node kon via OpsCenter vastgesteld worden dat er data uitgewisseld werd tussen de nodes.
Als men vervolgens de node, die eerst online was, uitschakelt, kan men toch de geüpdatete data terugvinden op de node die juist terug online komt.
Met het onderzoek dat in deze bachelorproef gedaan is, kan besloten worden dat Cassandra geen last heeft van ''single points of failure''.

Voor de back-ups kan besloten worden dat deze nog altijd nodig zijn.
De replica's zijn hier geen volwaardige vervangers voor.
Door het gebruik van replica's wordt de beschikbaarheid van de data gegarandeerd, maar dit is geen garantie tegen corrupte data.

Eén opmerking die bij dit alles moet gemaakt worden is dat alle test op virtuele machines zijn uitgevoerd en waarbij er slechts één datacenter beschikbaar was.
Door het gebruik van virtuele machines zijn de absolute tijdsgegevens in deze bachelorproef niet representatief voor een echte cluster waar de machines voor Cassandra alleen zijn voorbehouden.

\cite{kan2014cassandra} maakte reeds een mooie verwoording van wat Cassandra nu precies inhoud.
Niets in deze bachelorproef heeft het tegendeel kunnen bewijzen.

\emph{
	''Cassandra can be simply described in a single phrase: a massively scalable, highly available open source NoSQL database that is based on peer-to-peer architecture.''
}
\citep{kan2014cassandra}


\bibliographystyle{apa}
\bibliography{Bachelorproef-Lorenz-Verschingel}

%%---------- Back matter -------------------------------------------------

\listoffigures
\listoftables
\lstlistoflistings

\end{document}
